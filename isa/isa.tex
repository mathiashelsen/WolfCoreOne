\documentclass[a4paper,11pt]{article}


\usepackage{a4wide}

\usepackage{array}
\newcolumntype{L}[1]{>{\raggedright\let\newline\\\arraybackslash\hspace{0pt}}m{#1}}
\newcolumntype{C}[1]{>{\centering\let\newline\\\arraybackslash\hspace{0pt}}m{#1}}
\newcolumntype{R}[1]{>{\raggedleft\let\newline\\\arraybackslash\hspace{0pt}}m{#1}}

\begin{document}
\title{WolfCoreOne: ISA Specification v0.1}
\author{}
\date{\today}
\maketitle
PRELIMINARY -- WORKING DOCUMENT
\section{Introduction to the architecture}
The core makes use of a pipeline architecture:
\begin{enumerate}
    \item FETCH: In this stage the instruction is fetched and loaded
    \item DECODE: The instruction is decoded and the correct registers are wired into the ALU
    \item EXECUTE: The ALU executes the decoded instruction, together with the data that was provided
    \item WRITEBACK: Depending on the set conditions, the register(s) are updated
\end{enumerate}
In general any part of the pipeline is completely independent and agnostic of what the other
parts are doing. Of course, there are exceptions to this rule. The DECODE block for example, can schedule
(or multiplex) ALU output into ALU input. The ALU itself, is completely unaware where its input data is coming from.

\section{Introduction to the ISA}
Due to the pipeline nature of the architecture take into account the following:
\begin{itemize}
    \item When writing a value to a register, will only take effect on instructions delayed ... 
    \item When branching, the entire pipeline is hard flushed and no writeback can take place
\end{itemize}

\begin{table}
	\begin{center}
	\begin{tabular}{|C{8cm}|C{6cm}|C{2cm}|}
		\hline
		[7:4] & [3:1] & [0] \\ \hline 
		Register C & Writeback condition & Update local status register \\ \hline \hline
	\end{tabular}
	\begin{tabular}{|C{6cm}|C{2cm}|}
		\hline
		[15:13] & [12:8] \\ \hline
		LSB of Imm & Opcode selection bits \\ \hline \hline
	\end{tabular}
	\end{center}
\end{table}

\end{document}
